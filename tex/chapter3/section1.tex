%----------------------------------------------------------------------------
\section{Címkék és hivatkozások}
%----------------------------------------------------------------------------
A \LaTeX~dokumentumban címkéket (\verb+\label+) rendelhetünk ábrákhoz, táblázatokhoz, fejezetekhez, listákhoz, képletekhez stb. Ezekre a dokumentum bármely részében hivatkozhatunk, a hivatkozások automatikusan feloldásra kerülnek.

A sablonban makrókat definiáltunk a hivatkozások megkönnyítéséhez. Ennek megfelelően minden ábra (\emph{figure}) címkéje \verb+fig:+ kulcsszóval kezdődik, míg minden táblázat (\emph{table}), képlet (\emph{equation}), fejezet (\emph{section}) és lista (\emph{listing}) rendre a \verb+tab:+, \verb+eq:+, \verb+sect:+ és \verb+listing:+ kulcsszóval kezdődik, és a kulcsszavak után tetszőlegesen választott címke használható. Ha ezt a konvenciót betartjuk, akkor az előbbi objektumok számára rendre a \verb+\figref+, \verb+\tabref+, \verb+\eqref+, \verb+\sectref+ és \verb+\listref+ makrókkal hivatkozhatunk. A makrók paramétere a címke, amelyre hivatkozunk (a kulcsszó nélkül). Az összes említett hivatkozástípus, beleértve az \verb+\url+ kulcsszóval bevezetett web-hivatkozásokat is a  \verb+hyperref+\footnote{Segítségével a dokumentumban megjelenő hivatkozások nem csak dinamikussá válnak, de színezhetők is, bővebbet erről a csomag dokumentációjában találunk. Ez egyúttal egy példa lábjegyzet írására.} csomagnak köszönhetően aktívak a legtöbb PDF-nézegetőben, rájuk kattintva a dokumentum megfelelő oldalára ugrik a PDF-néző vagy a megfelelő linket megnyitja az alapértelmezett böngészővel. A \verb+hyperref+ csomag a kimeneti PDF-dokumentumba könyvjelzőket is készít a tartalomjegyzékből. Ez egy szintén aktív tartalomjegyzék, amelynek elemeire kattintva a nézegető behozza a kiválasztott fejezetet.
