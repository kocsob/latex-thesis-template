%----------------------------------------------------------------------------
\section{Képletek}
%----------------------------------------------------------------------------
Ha egy formula nem túlságosan hosszú, és nem akarjuk hivatkozni a szövegből, mint például a $e^{i\pi}+1=0$ képlet, \emph{szövegközi képletként} szokás leírni. Csak, hogy másik példát is lássunk, az $U_i=-d\Phi/dt$ Faraday-törvény a $\rot E=-\frac{dB}{dt}$ differenciális alakban adott Maxwell-egyenlet felületre vett integráljából vezethető le. Látható, hogy a \LaTeX-fordító a sorközöket betartja, így a szöveg szedése esztétikus marad szövegközi képletek használata esetén is.

Képletek esetén az általános konvenció, hogy a kisbetűk skalárt, a kis félkövér betűk ($\mathbf{v}$) oszlopvektort -- és ennek megfelelően $\mathbf{v}^T$ sorvektort -- a kapitális félkövér betűk ($\mathbf{V}$) mátrixot jelölnek. Ha ettől el szeretnénk térni, akkor az alkalmazni kívánt jelölésmódot célszerű külön alfejezetben definiálni. Ennek megfelelően, amennyiben $\mathbf{y}$ jelöli a mérések vektorát, $\mathbf{\vartheta}$ a paraméterek vektorát és $\hat{\mathbf{y}}=\mathbf{X}\vartheta$ a paraméterekben lineáris modellt, akkor a \emph{Least-Squares} értelemben optimális paraméterbecslő $\hat{\mathbf{\vartheta}}_{LS}=(\mathbf{X}^T\mathbf{X})^{-1}\mathbf{X}^T\mathbf{y}$ lesz.

Emellett kiemelt, sorszámozott képleteket is megadhatunk, ennél az \verb+equation+ és a \verb+eqnarray+ környezetek helyett a korszerűbb \verb+align+ környezet alkalmazását javasoljuk (több okból, különféle problémák elkerülése végett, amelyekre most nem térünk ki). Tehát
\begin{align}
\dot{\mathbf{x}}&=\mathbf{A}\mathbf{x}+\mathbf{B}\mathbf{u},\\
\mathbf{y}&=\mathbf{C}\mathbf{x},
\end{align}
ahol $\mathbf{x}$ az állapotvektor, $\mathbf{y}$ a mérések vektora és $\mathbf{A}$, $\mathbf{B}$ és $\mathbf{C}$ a rendszert leíró paramétermátrixok. Figyeljük meg, hogy a két egyenletben az egyenlőségjelek egymáshoz igazítva jelennek meg, mivel a mindkettőt az \& karakter előzi meg a kódban. Lehetőség van számozatlan kiemelt képlet használatára is, például
\begin{align}
\dot{\mathbf{x}}&=\mathbf{A}\mathbf{x}+\mathbf{B}\mathbf{u},\nonumber\\
\mathbf{y}&=\mathbf{C}\mathbf{x}\nonumber.
\end{align}
Mátrixok felírására az $\mathbf{A}\mathbf{x}=\mathbf{b}$ inhomogén lineáris egyenlet részletes kifejtésével mutatunk példát:
\begin{align}
\begin{bmatrix}
a_{11} & a_{12} & \dots & a_{1n}\\
a_{21} & a_{22} & \dots & a_{2n}\\
\vdots & \vdots & \ddots & \vdots\\
a_{m1} & a_{m2} & \dots & a_{mn}
\end{bmatrix}
\begin{pmatrix}x_1\\x_2\\\vdots\\x_n\end{pmatrix}=
\begin{pmatrix}b_1\\b_2\\\vdots\\b_m\end{pmatrix}.
\end{align}
A \verb+\frac+ utasítás hatékonyságát egy általános másodfokú tag átviteli függvényén keresztül mutatjuk be, azaz
\begin{align}
W(s)=\frac{A}{1+2T\xi s+s^2T^2}.
\end{align}
A matematikai mód minden szimbólumának és képességének a bemutatására természetesen itt nincs lehetőség, de gyors referenciaként hatékonyan használhatók a következő linkek:\\
\indent\url{http://www.artofproblemsolving.com/LaTeX/AoPS_L_GuideSym.php},\\
\indent\url{http://www.ctan.org/tex-archive/info/symbols/comprehensive/symbols-a4.pdf},\\
\indent\url{ftp://ftp.ams.org/pub/tex/doc/amsmath/short-math-guide.pdf}.\\
Ez pedig itt egy magyarázat, hogy miért érdemes \verb+align+ környezetet használni:\\
\indent\url{http://texblog.net/latex-archive/maths/eqnarray-align-environment/}.