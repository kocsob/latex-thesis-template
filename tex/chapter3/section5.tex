%----------------------------------------------------------------------------
\section{Irodalmi hivatkozások}\label{sect:HowtoReference}
%----------------------------------------------------------------------------
Egy \LaTeX dokumentumban az irodalmi hivatkozások definíciójának két módja van. Az egyik a \verb+\thebibliograhy+ környezet használata a dokumentum végén, az \verb+\end{document}+ lezárás előtt.
\begin{lstlisting}[frame=single,float=!ht]
\begin{thebibliography}{9}

\bibitem{Lamport94} Leslie Lamport, \emph{\LaTeX: A Document Preparation System}. 
Addison Wesley, Massachusetts, 2nd Edition, 1994.

\end{thebibliography}
\end{lstlisting}

Ezek után a dokumentumban a \verb+\cite{Lamport94}+ utasítással hivatkozhatunk a forrásra. A fenti megadás viszonylag kötetlen, a szerző maga formázza az irodalomjegyzéket. 

Egy sokkal professzionálisabb módszer a BiB\TeX~használata, ezért ez a sablon is ezt támogatja. Ebben az esetben egy külön szöveges adatbázisban definiáljuk a forrásmunkákat, és egy külön stílusfájl határozza meg az irodalomjegyzék kinézetét. Ez, összhangban azzal, hogy külön formátumkonvenció határozza meg a folyóirat-, a könyv-, a konferenciacikk- stb. hivatkozások kinézetét az irodalomjegyzékben (a sablon használata esetén ezzel nem is kell foglalkoznia a hallgatónak, de az eredményt célszerű ellenőrizni). A felhasznált hivatkozások adatbázisa egy \verb+.bib+ kiterjesztésű szöveges fájl, amelynek szerkezetét a \listref{Bibtex} kódrészlet demonstrálja. A forrásmunkák bevitelekor a sor végi vesszők külön figyelmet igényelnek, mert hiányuk a BiB\TeX-fordító hibaüzenetét eredményezi. A forrásmunkákat típus szerinti kulcsszó vezeti be (\verb+@book+ könyv, \verb+@inproceedings+ konferenciakiadványban megjelent cikk, \verb+@article+ folyóiratban megjelent cikk, \verb+@techreport+ valamelyik egyetem gondozásában megjelent műszaki tanulmány, \verb+@manual+ műszaki dokumentáció esetén stb.). Nemcsak a megjelenés stílusa, de a kötelezően megadandó mezők is típusról-típusra változnak. Egy jól használható referencia a \url{http://en.wikipedia.org/wiki/BibTeX} oldalon található.
\begin{lstlisting}[frame=single,float=!ht,caption=Példa szöveges irodalomjegyzék-adatbázisra BiBTeX használata esetén., label=listing:Bibtex]
@BOOK{Wettl04,
  author="Ferenc Wettl and Gyula Mayer and Péter Szabó",
  title="\LaTeX~kézikönyv",
  publisher="Panem Könyvkiadó",
  year=2004
}
@ARTICLE{Candy86,
  author ="James C. Candy",
  title  ="Decimation for Sigma Delta Modulation",
  journal="{IEEE} Trans.\ on Communications",
  volume =34,
  number =1,
  pages  ="72--76",
  month  =jan,
  year   =1986,
}
@INPROCEEDINGS{Lee87,
  author =       "Wai L. Lee and Charles G. Sodini",
  title =        "A Topology for Higher Order Interpolative Coders",
  booktitle =    "Proc.\ of the IEEE International Symposium on 
  Circuits and Systems",
  year =         1987,
  vol =          2,
  month =        may # "~4--7",
  address =      "Philadelphia, PA, USA",
  pages =        "459--462"
}
@PHDTHESIS{KissPhD,
  author =   "Peter Kiss",
  title =    "Adaptive Digital Compensation of Analog Circuit Imperfections 
  for Cascaded Delta-Sigma Analog-to-Digital Converters",
  school =   "Technical University of Timi\c{s}oara, Romania",
  month =    apr,
  year =     2000
}
@MANUAL{Schreier00,
  author = "Richard Schreier",
  title  = "The Delta-Sigma Toolbox v5.2",
  organization = "Oregon State University",
  year   = 2000,
  month  = jan,
  note   ="\newline URL: http://www.mathworks.com/matlabcentral/fileexchange/"
}
@MISC{DipPortal,
	author="Budapesti {M}űszaki és {G}azdaságtudományi {E}gyetem 
	{V}illamosmérnöki és {I}nformatikai {K}ar",
  title="{D}iplomaterv portál (2011 február 26.)",
  howpublished="\url{http://diplomaterv.vik.bme.hu/}",
}}
\end{lstlisting}

A stílusfájl egy \verb+.sty+ kiterjesztésű fájl, de ezzel lényegében nem kell foglalkozni, mert vannak beépített stílusok, amelyek jól használhatók. Ez a sablon a BiB\TeX-et használja, a hozzá tartozó adatbázisfájl a \verb+mybib.bib+ fájl. Megfigyelhető, hogy az irodalomjegyzéket a dokumentum végére (a \verb+\end{document}+ utasítás elé) beillesztett \verb+\bibliography{mybib}+ utasítással hozhatjuk létre, a stílusát pedig ugyanitt a  \verb+\bibliographystyle{plain}+ utasítással adhatjuk meg. Ebben az esetben a \verb+plain+ előre definiált stílust használjuk (a sablonban is ezt állítottuk be). A \verb+plain+ stíluson kívül természetesen számtalan más előre definiált stílus is létezik. Mivel a \verb+.bib+ adatbázisban ezeket megadtuk, a BiB\TeX-fordító is meg tudja különböztetni a szerzőt a címtől és a kiadótól, és ez alapján automatikusan generálódik az irodalomjegyzék a stílusfájl által meghatározott stílusban.

Az egyes forrásmunkákra a szövegből továbbra is a \verb+\cite+ paranccsal tudunk hivatkozni, így a \listref{Bibtex} kódrészlet esetén a hivatkozások rendre \verb+\cite{Wettl04}+, \verb+\cite{Candy86}+, \verb+\cite{Lee87}+, \verb+\cite{KissPhD}+, \verb+\cite{Schreirer00}+ és \verb+\cite{DipPortal}+. Az irodalomjegyzékben alapértelmezésben csak azok a forrásmunkák jelennek meg, amelyekre található hivatkozás a szövegben, és ez így alapvetően helyes is, hiszen olyan forrásmunkákat nem illik az irodalomjegyzékbe írni, amelyekre nincs hivatkozás.

Mivel a fordítási folyamat során több lépésben oldódnak fel a szimbólumok, ezért gyakran többször (TeXLive és TeXnicCenter esetén 2-3-szor) is le kell fordítani a dokumentumot. Ilyenkor ez első 1-2 fordítás esetleg szimbólum-feloldásra vonatkozó figyelmeztető üzenettel zárul. Ha hibaüzenettel zárul bármelyik fordítás, akkor nincs értelme megismételni, hanem a hibát kell megkeresni. A \verb+.bib+ fájl megváltoztatáskor sokszor nincs hatása a változtatásnak azonnal, mivel nem mindig fut újra a BibTeX fordító. Ezért célszerű a változtatás után azt manuálisan is lefuttatni (TeXnicCenter esetén \verb+Build/BibTeX+).

Hogy a szövegbe ágyazott hivatkozások kinézetét demonstráljuk, itt most sorban meghivatkozzuk a \cite{Wettl04}, \cite{Candy86}, \cite{Lee87}, \cite{KissPhD} és az \cite{Schreier00} forrásmunkát, valamint az \cite{DipPortal} weboldalt.

Megjegyzendő, hogy az ékezetes magyar betűket is tartalmazó \verb+.bib+ fájl az \verb+inputenc+ csomaggal betöltött \verb+latin2+ betűkészlet miatt fordítható. Ugyanez a \verb+.bib+ fájl hibaüzenettel fordul egy olyan dokumentumban, ami nem tartalmazza a \verb+\usepackage[latin2]{inputenc}+ sort. Speciális igény esetén az irodalmi adatbázis általánosabb érvényűvé tehető, ha az ékezetes betűket speciális latex karakterekkel helyettesítjük a \verb+.bib+ fájlban, pl. á helyett \verb+\'{a}+-t vagy ő helyett \verb+\H{o}+-t írunk. 

Oldaltörés következik (ld. forrás).
\newpage